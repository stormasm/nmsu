\documentstyle[12pt]{report}
\parskip 0.20in
\textheight 8.75in
\textwidth 6.0in
\topmargin -0.25in
\oddsidemargin 0.40in

\begin{document}
\baselineskip 0.30in

\setcounter{chapter}{6}
\setcounter{page}{40}

\chapter{Conclusions and Final Comments}	

	The results presented in this paper are over all not very
optimistic for someone who is interested in solving the AI problem in
the near future.  The point being that although these techniques are
able to learn an input output mapping fairly well, the techniques fail
miserably when it comes to generalization.  It might just be that the
approach for solving the AI problem is incorrect.  The algorithms
presented in this paper are intended to be general in a sense that the
same algorithm is trying to solve many different problems.  This
philosophy is of course motivated by the neural network architecture
which is intended to be mimicking the brain.  At least this was the
reasoning that the inventors or these algorithms put forth in their
early articles and still to this day.

	One of the major criticisms of these techniques is why do
addition or multiplication in such a round about way especially when
it can be done in silicon in microseconds.  The same argument can be
presented for all the problems presented in this paper.  An input
string can be parsed and the numbers of ones counted a lot quicker
than the time it takes to fit a surface to a boolean hypercube.
Although these criticisms may be valid, the point of the paper is not
to analyze individual algorithms.

\section{What size problem gives the best generalization ?}

	This question is one that seemed important from the onset of
the research but in thinking about the problem more it is clear that
this issue has a fairly simple answer.  Many have argued that in order
to get statistically significant results that large problems have to
be analyzed.  This principle is not as important as the percentage of
information that is used for generalization.  For example, if there
are sixteen elements in the problem set and one does generalization on
fifty percent of the set; will the results be similar if there are one
thousand elements in the training set and generalization is done on
five hundred elements.  This question is important and needs some
further research done to answer it convincingly.

\section{And finally...}

	One of the big open questions of research in the area of
learning algorithms is to be able to classify discrete problems.  One
should be able to look at a particular type of problem and then
determine which learning algorithm is best suited for the task at
hand.

	Ideally, with many studies completed, the research community
will begin to understand which methods are suitable for a certain
class of problems.  Clearly being able to group discrete types of
problems into classes would be a major step forward in the field of
learning algorithms.  

	Examples of this type of idea can has been studied by Wolfram
[1986] in determining what types of dynamical systems give rise to
Cellular Automata, and the problems studied by Garey and Johnson [1979]
in grouping classes of problems in the NP class.

	Just touching the surface of this idea would be a major
endeavor especially in the area of simple discrete problems and
certainly an open research question.  In this paper, it is shown that
for simple binary encoded problems, it really does not matter which
technique one uses for {\it accuracy} of learning.  However, the local
linear techniques are orders of magnitude quicker in run time
efficiency.  The speed result is enough of a reason for more people to
consider using local linear techniques when considering which learning
algorithm will best suit their need.  Especially after realizing that
mathematically both techniques are basically doing surface fitting.

\end{document}
