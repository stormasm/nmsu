\documentstyle[12pt]{report}
\parskip 0.20in
\textheight 8.75in
\textwidth 6.0in
\topmargin -0.25in
\oddsidemargin 0.40in

\begin{document}
\baselineskip 0.30in

\setcounter{chapter}{1}
\setcounter{page}{4}
\chapter{Back Propagation}	


	Back propagation is the most common learning algorithm in use
today.  Its popularity started mainly from a paper published by
Rumelhart, Hinton, and Williams [1986].  The data structure that back
propagation acts on is called a neural network.  The idea of neural
networks was introduced by McCulloch and Pitts [1943].

	One of the major motivations for having back propagation as
one of the learning algorithms is to use it as a bench mark against
other learning algorithms which are not nearly as common.  Because
back propagation is so common, the experimental results can be much
better understood and validated.  This practice is very common in any
type of research on algorithms.  Researchers take the best known
algorithm that does something, and shows how the new algorithm is
superior for one reason or another.

	From a simple point of view, a neural network is a layered
graph with a set of input nodes, hidden nodes, and output nodes.  The
nodes are connected by a set of weights.  The connectivity of the
network is purely arbitrary depending on the problem.  The basic
question of how to connect the nodes is an open research problem.  In
all the examples in this paper the input units will be connected to
the hidden units and the hidden units will be connected to the output
units.

\section{Number of Input Units and Output Units}

	The straight forward aspect of building a neural network is
calculating the number of input units and output units.  This can
easily be determined by looking at the constraints on the problem.
For example, if you want to input integers in binary form that have a
range of 0 to 240, then you need eight input units to represent all
possible values.  The same analogy holds true for the number of output
units.  If the problem is addition and two numbers are being added
then the maximum number of output units to represent the largest value
is chosen.  The same is true for multiplication.  If the problem is to
count the number of bit positions in a five bit string, then three
output units would be sufficient to represent the number five.  The
parity problem only requires one output unit since the answer is
simply only one or zero.

\section{The Number of Hidden Units}

	The tricky part to designing a neural network is coming up
with the number of hidden nodes.  Up till now, the traditional method
for overcoming this problem has been trial and error.  However, for
some very restricted types of boolean valued problems some methods
have been devised by Baum [1985] to determine the minimum number of
hidden nodes needed to solve a particular problem.  

	The number of hidden units to choose for a particular problem
is an open area of research, but in these experiments the number of
hidden units was determined by trial and error.  A number was chosen
that appeared to be a bit large, and then back propagation was run on
that particular number of hidden units.  If the network learned the
function perfectly, then a smaller number was chosen and back
propagation was run again.  This procedure was repeated until the
minimum number of hidden units was found which would learn the input
output mapping perfectly.  This is a reasonable method for determining
the number of hidden units and also the most common method in use
today.

	Using the optimal number of hidden units is very significant
because if you use too many hidden units the network will {\it over
learn} the problem.  This is not so much a factor when one is just
trying to get a network to learn an input output mapping, but the
optimal network truly comes into play when the generalization phase
arises.  Even if there are too many hidden units in a network, chances
are the network will learn the input output mapping, but the
generalization score will go way down.  There are advantages to having
more than the optimal number of hidden units in a network.  One of the
big advantages is that the network will converge a lot faster meaning
that it won't take the network nearly as long to learn a particular
function.  When dealing with big problems, the difference in
convergence time could be several days or even weeks.  So sometimes
for starters people will begin with an over abundance of hidden units
just to see if the network can learn a particular input output
mapping.  It is also a good first approximation to over estimate the
number of hidden units a network will use.

\section{Topology of the Weight Space}

	The topology of the weight space has not yet been discussed in
this paper and it is worth mentioning the analogy that is depicted in
the literature.  The weight space is typically equated to a mountain
range with steep narrow canyons and high mountain peaks.  Ideally, if
one was trying to find the highest peak in the range they would simply
get in an airplane and fly above the range and point out the highest
peak.  However, the search in neural networks is very local and
basically the algorithm can only see one step in front of itself.
This causes many problems, especially since one never knows if they
are at the lowest point in the range.  All these ideas relate to
optimization theory and function approximation which have been studied
for quite a few years in mathematics.
   
\section{The Learning Rate}
   
	Another major factor in the study of neural networks is the
learning rate.  The learning rate determines how fast a particular
network will converge to a solution.  In some cases, it is
advantageous to converge slowly to a particular solution in order to
avoid getting trapped in a local minimum.  A learning algorithm called
{\it simulated annealing} does fairly well at finding a global minimum
but the disadvantage is that the algorithm is very slow.  Associated
with the learning rate is a another term called the momentum.  It is
used to accelerate the learning rate in certain areas of the weight
space.  Topologically it helps get out of long ravines which cause
oscillations in the weight space.  The momentum term helps push the
search out of steep narrow gullys which are usually local minima.

\section{The Algorithm}

	Back propagation is an algorithm which fits a multi
dimensional surface or function to a set of input output pairs.  The
values of the weights in the neural network defines the multi
dimensional surface.  The algorithm is intuitively very simple.  Given
a set of input output pairs, adjust the weights or surface so that all
pairs will be satisfied to a certain accuracy delta.  One way to write
this algorithm would be to vary all the weights in some ordered manner
until you hit on the proper combination of weights.  This method would
eventually find a solution if one exists, however, from a
combinatorial point of view it is not very efficient.

	There are two main components to this algorithm which can be
thought of as the data structures.  One main component is the set of
weights in the network and the other component is the set of input
output pairs which the network is trying to learn.  The over all
principle of the algorithm is fairly simple.  Given a particular input
pattern, feed or propagate those values through the network until it
has reached an output layer.  Once reaching the output layer, then
compare the values of the output layer with the {\it teaching} or {\it
target} layer.  The difference between the actual value and the {\it
teaching} value are then propagated back through the network.  In
other words, the exact opposite set of steps is performed going
backwards through the network as opposed to forwards through the
network.  Once, the {\it forward} and {\it backward} pass have been
completed then the set of weights are updated to reflect how the
network has changed in that particular pass.  

\subsection{A Detailed Example of Back Propagation}

	The easiest way to explain back progation is to step through a
very simple example which depicts exactly how the algorithm works.
{\it Figure 1} shows that there are three layers in the network; two
units in the input layer denoted 1 and 2, one unit in the hidden layer
denoted 3, and two units in the output layer denoted 4 and 5.  Also
note that unit 1 is connected to unit 3 via a weight labeled {\it a},
unit 2 is connected to unit 3 via a weight labeled {\it b}, unit 3 is
connected to unit 4 via a weight labeled {\it c} and unit 3 is
connected to unit 5 via a weight labeled {\it d}.  The terminology
used for this network structure is fully connected, in other words all
units in the input layers are connected to units in the hidden layers,
and all units in the hidden layers are connected to units in the
output layers.  In all the examples in this paper the networks are
fully connected, but for certain problems only specific units are
connected to each other.

	The first step to the back prop algorithm is to initialize all
the weights to small random values between $-0.30$ and $0.30$.  For
this particular example let 

	weight $a = .1$, 

	weight $b = .2$, 

	weight $c = .3$ 

	weight $d = .2$

	Let the input output mapping for this particular problem be (0
1 yields 1 0).  So, the initial input for 

	input $unit 1 = 0$, 

	input $unit 2 = 1$,

        target $unit 4 = 1$ 

	target $unit 5 = 0$.

	Now that the all the weights and units are initialized the
first thing that happens is called the forward pass.  The calculations
for this subroutine proceeds as follows.

	$Unit 1 * Weight a = 0 * .1 = 0$

	$Unit 2 * Weight b = 1 * .2 = .2$

	We are now able to calculate the input value for unit 3 which
is simply the sum of the above two calculations.  However, there is
one further twist which is not obvious and that is that once the sum
is calculated then the {\it sigmoid} of the sum is the final answer.

	The sigmoid function is defined to be

	$$1 \over {1 + e ^ {-x}}$$

	Where x represents the sum of all the inputs coming into a
particular unit.  In this case $x = 0 + 0.2$

	There are some special properties of the sigmoid function
which is worth mentioning.  The output of the sigmoid function will
always be between zero and one independent of the size of the sum of
the units.  The following table is a rough outline of sigmoid values.

	$sigmoid(-3.00) = 0.05$

	$sigmoid(-2.00) = 0.12$

	$sigmoid(-1.00) = 0.27$

	$sigmoid( 0.00) = 0.50$

	$sigmoid( 1.00) = 0.73$

	$sigmoid( 2.00) = 0.88$

	$sigmoid( 3.00) = 0.95$

	So, the new calculated input value for unit 3 is

	$sigmoid(0 + .200) = .55$.  

	The next step is to continue to propagate the newly found
input value from the hidden layer to the output layer.

	$Unit 3 * Weight c = .55 * .3 = .165$

	$Unit 3 * Weight d = .55 * .2 = .110$

	Now we take the sum and calculate the sigmoid value to come up
with new input values for units 4 and 5.

	$unit 4 = sigmoid(.165) = .54$

	$unit 5 = sigmoid(.110) = .53$

	At this point in time we have completed the forward pass.  In
the case of the {\it delivery phase} which is discussed in chapter 6
this would be the end of the algorithm.  However, since we are in the
training phase we need to do several more things.  The details of the
rest of the algorithm will not be discussed in this paper due to the
fact that the derivation of the formulas are based upon advanced
calculus.  However, the general idea of what happens will be outlined.

	Once the forward pass is completed, the error signal is
calculated.  The calculation of the error signal is based on looking
at the current values which are (.54, .53) and the target values
which are (1 0).  The difference between these values is propagated
back through the network in a similar fashion to the forward pass.
This phase of the algorithm is called the backward pass.  Once the
backward pass is complete the weights of the network are changed to
reflect the updating of the information.  

	The algorithm starts over with the forward pass immediately
after the weights have been changed.  This method is continued until
the algorithm has adjusted the weights so that when the input values
are propagated through the network the {\it expected or teaching}
values are the outputs.

\section{The Final Network}

	After training the network, it just may happen that some of
the weights are near zero. If this does happen, that particular
connection is very weak and does not play a major role in the over all
network structure.  In general, the higher the value of the weights in
the network the more active role that particular set of nodes plays in
the problem.

\section{The Mathematics of Back Propagation}

	The original version of this paper contained a derivation of
the back propagation mathematics.  However, in this newer version the
mathematics of back propagation has been intentionally omitted because
an understanding of advanced calculus is required to fully understand
the derivation of this technique.  It has been determined that for the
audience of this paper, they can do further research in this area by
simply reading the article in [McClelland 1986].  The mathematics of
chapter three has been intentionally left in since it only deals with
basic linear algebra.  This point is also essential in understanding
why the local linear method is simpler and much more intuitive.

\end{document}
