\documentstyle[12pt]{report}
\parskip 0.20in
\textheight 8.75in
\textwidth 6.0in
\topmargin -0.25in
\oddsidemargin 0.40in

\begin{document}
\baselineskip 0.30in

\setcounter{chapter}{2}
\setcounter{page}{13}

\chapter{Local Linear Techniques}	

	Although local linear techniques are not nearly as popular as
back propagation, they are actually a superior learning algorithm for
two main reasons.  First and foremost, they are orders of magnitude
faster in execution time.  Secondly, they are much easier to
understand from a mathematical point of view.

	Local linear techniques are based on a simple information
theoretic concept stating that data points which are close to each
other are more significant than points which are further away.  The
main idea is given a question; find an answer to the question by
simply looking at the neighbors nearest to the question.  The fact
that makes these methods difficult to picture mentally is that most of
the input domains of even simple problems have high dimensional spaces
associated with them.  A hyperplane is defined to be a surface which
lives in an $n$ dimensional space.  The mathematics behind these
algorithms is based simply on solving linear equations, which most
people learn in high school algebra.

	Local linear techniques have been used almost exclusively in
the past to analyze time series [Farmer].  When analyzing a time
series, the idea is that the nearest neighbors contain the most
information relative to the data point in question.  This intuition
has been proven by experiment in a sense that the best prediction
methods to date have been local linear techniques and back
propagation.  There is still debate focused on which technique is
better [Farmer, Lapedes].  In other words, both techniques have
performed about the same on a given time series.

 	A time series is a set of data points with some arbitrary
measure of time on the x axis and the value at that point of time on
the y axis.  Examples of time series include the daily prices of a
particular stock, sun spots, and Las Cruces weather temperatures in
1989 on a daily basis.

\section{Local Linear Algorithms}
	
	There are many different algorithms which are derivatives of
the local linear method.  I have chosen to analyze three of these
methods simply to show that the idea of nearest neighbors are
significant from an information theoretic point of view.  The only
difference between the following three algorithms is the method in
which singularities are resolved.

\subsection{Local Linear}

	The easiest way to see how this algorithm works is to step
through a very simple example in three dimensions.  Since we are
dealing with binary encoded data, there are only eight ways to
represent three dimensional data.  Lets assume that seven of the eight
possible representations are available.  They are :

\begin{center}
\begin{tabular}{llllcc}
\multicolumn{1}{c}{Case} &
\multicolumn{3}{c}{Input} &
\multicolumn{1}{c}{Output} &
\multicolumn{1}{c}{Hamming Distance} \\
0. & 0 & 0 & 0 & 0 & 1 \\
1. & 0 & 0 & 1 & 0 & 2 \\
2. & 0 & 1 & 1 & 1 & 1 \\
3. & 1 & 0 & 0 & 0 & 2 \\
4. & 1 & 0 & 1 & 0 & 3 \\
5. & 1 & 1 & 0 & 1 & 1 \\
6. & 1 & 1 & 1 & 1 & 2 
\end{tabular}
\end{center}

	The above function represents the idea of {\bf outputting the
middle bit position}.  The case that will be generalized on is $(0, 1,
0)$ which is not in the training set.
	
	The idea behind learning algorithms is to train itself on a
subset of the data and then to generalize to the missing pieces.  The
missing piece in this puzzle is the the question (0 1 0). The correct
answer to this question is (1), which is the middle bit position of
the input space.

	The first thing the algorithm does is determine the hamming
distance between the question which is (0 1 0) and the other inputs.
Note the hamming distances between the question and the other
neighbors in the input space in the above table.

	After calculating the hamming distances, the distances are
ordered according to the hamming distance to the question.  Points
that have distances closer to the question are more important than
points further away.  With this in mind the following ordering is
calculated (5 2 0 6 3 1 4).  Note that although some of the distances
are the same, I just randomly start from the bottom and go up the list.

	The number of significant points that are used to calculate
the simplex is the number of input dimensions plus one.  So in this
particular case I use 3 + 1 = 4 points in the simplex.  The four
points in the simplex are just simply the first four points in the
above list.  After determining the points in the simplex, I solve the
set of four equations and four unknowns.  

      	(5)    $1w + 1x + 0y + z = 1$
      
      	(2)    $0w + 1x + 1y + z = 1$      

      	(0)    $0w + 0x + 0y + z = 0$ 

      	(6)    $1w + 1x + 1y + z = 1$

       	The values for the variables are as follows :

               w = 0,  x = 1,  y = 0,  z = 0

        Then plug into the equation which is not part of training set.

          $0w + 1x + 0y + z = $

          $0 + 1 + 0 + 0 = 1$

        So, as to be expected the answer is one (1)...

	After solving these equations one of two things can happen.
First, I can get back a unique solution to this set of equations and
therefore output an answer to the question.  Which is what happened
with the above example.  However, in a lot of cases the four equations
are not independent and therefore the algorithm returns the fact that
it has found a singularity.  If this happens, then the local linear
algorithm simply goes to the next nearest point and solves the
equations again.  It does this until a unique solution is found.


\subsection{Herbie II}

	The original Herbie algorithm was developed by David Wolpert
in an attempt to come up with a quick learning method that uses local
linear ideas.  After using his algorithm for awhile, I came up with a
derivative of Herbie which is suitable for simple binary encoded
problems.  Out of respect for David Wolpert, I decided to call my
algorithm Herbie II.  For a detailed description of the original
Herbie algorithm see Wolpert [1990] {\it ``A Benchmark for How Well
Neural Nets Generalize''}.

	The only difference between the above algorithm, {\it Local
Linear}, and the Herbie algorithm is what happens if a singularity is
found after solving the set of independent equations.  In the {\it
Local Linear} algorithm the next nearest neighbor was chosen and replaced
the point furthest away in the current equations.  The new set of
equations are then solved for a unique set of coefficients.

	The herbie algorithm takes a slightly different approach if a
singularity is found initially.  It also throws away the furthest
point just like local linear, but instead of taking the next closest
point, it creates a new point which lies on the normal to the
hyperplane.  After solving the set of linear equations, if there is
still a singularity then it uses an algorithm called {\it Weighted
Averaging} to find a solution.	

\subsection{Weighted Averaging}

	This technique is used as the final step in the {\it Herbie
algorithm}.  However, it can also be used as a stand alone learning
technique.  In the case of binary encoded problems, this algorithm
does not perform as well as the local linear method or {\it Herbie};
however, for predicting time series this algorithm is a very viable
technique.

	The example used to explain this technique is exactly the same
example that was used in section 3.1.1.  The algorithm is based on the
following two equations.  Note that $i$ is the index of a particular
point based on its hamming distance from the question.  So, a point
whose index value is $0$ is the point which is closest to the
question, and a point whose index value is $1$ is the point which is
second closest to the question.  The number of points in the simplex
equals $N$ which is the number of input dimensions plus one.

\vfill
\eject
	$$X = {\sum_{i = 0}^N} {output[i] \over
hammingdistance[i]}$$

	$$Y = {\sum_{i = 0}^N} {1 \over hammingdistance[i]}$$

	and the solution to the problem is

	$$answer = {X \over Y}$$

	So, in the case of the example in section 3.1.1 the hamming
distance indexes are $(5, 2, 0, 6, 3, 1, 4)$.  Only four of the seven
points are used since the number of input dimensions is three.

\begin{center}
\begin{tabular}{llllcc}
\multicolumn{1}{c}{Case} &
\multicolumn{3}{c}{Input} &
\multicolumn{1}{c}{Output} &
\multicolumn{1}{c}{Hamming Distance} \\
0. & 0 & 0 & 0 & 0 & 1 \\
1. & 0 & 0 & 1 & 0 & 2 \\
2. & 0 & 1 & 1 & 1 & 1 \\
3. & 1 & 0 & 0 & 0 & 2 \\
4. & 1 & 0 & 1 & 0 & 3 \\
5. & 1 & 1 & 0 & 1 & 1 \\
6. & 1 & 1 & 1 & 1 & 2 
\end{tabular}
\end{center}

$$X = {1 \over 1} + {1 \over 1} + {0 \over 1} + {1 \over 2} = 2.5$$

$$Y = {1 \over 1} + {1 \over 1} + { 1 \over 1} + {1 \over 2} = 3.5$$

	and the answer is
	$$answer = {X \over Y} = .71$$ 

	rounding gives the correct answer of $1$.
	
\end{document}
