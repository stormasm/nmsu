\documentstyle[12pt]{report}
\parskip 0.20in
\textheight 8.75in
\textwidth 6.0in
\topmargin -0.25in
\oddsidemargin 0.40in

\begin{document}
\baselineskip 0.30in

\chapter{Introduction}
		
	The purpose of this paper is to compare and contrast two types
of learning algorithms and how they perform on simple binary encoded
problems.  The first type of learning algorithm is a local linear
technique based on nearest neighbors in a multi dimensional simplex.
Three different algorithms based on this idea will be analyzed.  The
second type of learning algorithm will be the traditional back
propagation.  Simple binary encoded examples are chosen in order to to
understand the techniques clearly, rather than getting confused by
complicated learning domains.  Emphasis will be placed on showing
empirically how back propagation and local linear methods are both
doing surface fitting.  Since the experimental results for both
algorithms are basically the same, this will also delineate the fact
that both techniques are similar in {\it learning abilities}.

\section{What is a Learning Algorithm ?}	

	The term learning algorithm was coined by AI researchers in
the 1960's hoping to get computer programs to perform simple human
tasks.  Since that time, many different types of algorithms have
become popular.  In the 1960's Newell and Simon worked on the General
Problem Solver.  In the 1970's expert systems were used as a research
tool and then further developed into marketable products in the
1980's.  In the past five years, a third wave of algorithms based on
{\it connectionism} or neural networks have been popularized.

	It has not been made clear in the literature that the {\it
connectionist} algorithms are based upon traditional function
approximation or surface fitting techniques.  Given a set of points on
the x - y axis, a learning algorithm can use many different types of
methods to come up with the best fit.  They include least squares,
polynomial interpolation, splines, fourier analysis, back propagation
and local linear methods.  The technique that works best is based upon
how well the function makes predictions or generalizations.

	Given a set of fifty points on the x - y axis, one of the
above methods will come up with the best function.  That function is
the one which most accurately predicts the next ten points which were
not part of the training set.  This method of generalization can also
be called learning.

	In this paper, the domain and range of the functions are not
part of the x - y axis, but rather they are located on the boolean
hypercube.  Problems which are located on the hypercube are sometimes
called discretized problems.  These are opposite from analog or
continuous problems which most function approximations are used to
dealing with.  The following chapters outline two main types of
function approximators, namely back propagation and local linear
methods.

\section{Previous Work}

	The idea of local linear methods was first introduced by
Stephen Omohundro [1987] in a paper entitled "Efficient Algorithms
with Neural Network Behavior".  The idea of using nearest neighbors
for pattern classification was introduced by Cover [1967].
Omohundro's paper is a landmark paper and should be the first
reference read by anyone interested in further reading on this
subject.  In his paper he emphasizes the point that much more
efficient learning algorithms than back propagation are available, and
describes them from a theoretical point of view. He addresses a much
larger question than just the algorithm efficiency.  Omohundro tries to
group continuous and discrete problems into classes of problems and
then justifies why those types of problems are best suited for a
particular algorithm.  One crucial point in his paper deals with the
significance of pseudo random number generators and how they play a
very important role in the modern day stochastic simulations.  The one
minor criticism of Omohundro's work is the fact that he does not
actually run any experiments to test his ideas.

	Farmer and Sidorowich [1988] expand on Omohundro's ideas and
discuss prediction of chaotic time series.  Lapedes and Farber [1988]
study the same times series as Farmer but use neural networks to
analyze the data.  No specific comparisons of the two techniques are
outlined in either paper but the reader could certainly come up with a
better intuitive feel of analyzing continuous data using these
methods.

	David Wolpert [1989] is the first person to actually compare
both back propagation and local linear methods on discrete problems.
In this paper, he runs the local linear methods on a set of problems
and then compares them to the method of back propagation.  Wolpert
[1990b] is an excellent theoretical introduction to the ideas of local
linear methods.  Wolpert [1990a] compares the Sejnowski and Rosenberg
[1987] work on Net Talk, "Parallel Networks that Learn to Pronounce
English text" against local linear methods and shows that local linear
methods actually do better than back propagation.

	In this paper, I expand on the ideas of Wolpert's work by
choosing new binary encoded problems and verify empirically that both
back propagation and local linear methods perform about the same.  It
is hopefull that further research in this area will clarify which
methods are appropriate for a certain class of discrete problems.

\end{document}
