\documentstyle[12pt]{report}
\parskip 0.20in
\textheight 8.75in
\textwidth 6.0in
\topmargin -0.25in
\oddsidemargin 0.40in
\appendix
\begin{document}
\baselineskip 0.30in

\setcounter{page}{50}

\chapter{}
\section{Gradient Descent}

	Gradient Descent is a method for finding the maximum rate of
change at a particular point on a multidimensional surface.  In the
case of neural networks, the surfaces of interest usually are high in
dimension which means that they have many dependent variables.  The
concept in calculus of partial derivatives enables one to find the
rate of change with respect to a single variable.  The gradient is
defined to be the sum of all the partial derivatives which points in
the direction of maximum change.  For a detailed discussion of this
subject see Appendix B.

\section{Method of Least Squares}
		
	Since back propagation is very similar to the method of
least squares, it will be explained in order to convey and justify the
derivation of the back propagation algorithm.  

	The method of least squares is a technique that is used to fit
the best function to a set of observational data.  Since this data has
been observed, associated with the data is a small amount of noise.
In the case of this algorithm, the experimenter usually has a general
idea of what the function looks like.  For example, the function may
be a combination of simple linear functions, however the coefficients
of the function are unknown.

	$$f(x) = c_1 f_1(x)\ + \ c_2 f_2(x)\ + \ ..... \  + c_m f_m(x)$$

	The method of determining how well a particular function fits
a set of data is calculated by adding up the squares of the errors at
each of the observation points.  The goal is to minimize $E$ which is
defined to be the error.  

	$$E = \sum_{j = 1}^N (f(x_j) - y_j)^2$$

	For a detailed discussion of this subject see Appendix C.

\subsection{The Notation}

	The number of input output patterns will vary from problem to
problem. In the case of adding two three bit numbers then the total
number of input patterns is two to the sixth power which is 64.
To delineate input output pattern number 3 from pattern number
59 each set of input output pairs will be denoted by the letter {\bf p}.
The algorithm is based on layered feedforward networks and to
distinguish between the two layers we use two letters {\bf i} and {\bf
j}.  Layer {\bf i} is always the layer immediately below layer {\bf
j}.  Given any two layers, we could be looking at the fourth unit in
one particular layer $i = 4$ and the sixth unit in the layer
directly above it so $j = 6$.

	For each particular pattern there is an input pattern and an
output pattern.  The output pattern will be denoted by a {\bf t} where
the {\bf t} stands for the {\bf teaching pattern} or {\bf target
pattern}.  The other letter that is used in the notation is the letter
{\bf o}.  This letter delineates the output pattern that is actually
generated by the network at any particular layer.  The letter {\bf i}
is used only when dealing with an actual input unit.

	So, the following examples will delineate the above ideas.

	$t_{pj}$, $p = 3$, $j = 4$  The fourth teacher unit in pattern 3.

        $o_{pj}$, $p = 9$, $j = 2$  The second output unit in pattern 9.

        $o_{pi}$, $p = 9$, $i = 1$  The first input unit in pattern 9.

	Note once again, that the {\bf i's} and {\bf j's} strictly
distinguish relative ordering of particular layers in a network.
So, when one sees an {\bf i} they know that it means the layer below a
particular layer where the {\bf j} is referenced.

\subsection{The Sigmoid Function}	

	During the forward pass the sum of all the weights coming into
a particular unit are calculated.  However, this is not the value that
is stored as the input of that particular unit.  Instead, the sigmoid
of that particular value is stored.  The sigmoid function is defined as

	$$1 \over {1 + e^ {-(\sum_i \ w_{ji} \ o_{pi})}}$$

	The derivative of this particular function with respect to its
total input $net_{pj}$ is given by

	$${\partial o_{pj} \over \partial net_{pj}} \ = \ o_{pj} \ (1 - o_{pj})$$

\subsection{The Algorithm}

	This section will describe the rule for changing the weights
in the network.  The derivation of the rule is based on two main
principles, namely gradient descent and the method of least squares.
The key point is that the set of weights in the network defines the
actual function that is trying to be learned based upon a set of input
output patterns.  By minimizing the error surface through each pass,
one hopefully gets closer to the actual function.

	The formula for minimizing the error is exactly the same as
the one described in the section on least squares.  

	$$E_p \ = \ {1 \over 2} \sum_j \ {(t_{pj} - o_{pj}}) ^ 2$$

	Note, that $t_{pj}$ is a particular unit in the teacher
pattern which is equivalent to the {\it observed value} $y_j$.
$o_{pj}$ is a particular unit in the actual output pattern, derived by
the forward pass, which is equivalent to {\it the function value}
$f(x_j)$.  The $1 \over 2$ is strictly placed here for convenience.

	We now need to determine how the error measure changes with
respect to each weight.  We minimize E by taking a derivative of the
error surface with respect to each weight.  Note that $t_{pj}$ is
constant and does not change.
	
	$$\Delta w_{ji} = - \eta {\partial E_p \over \partial w_{ji}} =
- \eta \sum_j ({t_{pj} - o_{pj}}) {\partial o_{pj} \over \partial w_{ji}}$$


	Thus the formula for changing weights is based on the idea of
gradient descent.  This idea will be expanded in a future paper which
deals specifically with the ideas of least squares, directional
derivatives and back propagation.  Also, please check PDP chapter
eight for a complete derivation.

\end{document}
