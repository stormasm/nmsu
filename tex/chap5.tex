\documentstyle[12pt]{report}
\parskip 0.20in
\textheight 8.75in
\textwidth 6.0in
\topmargin -0.25in
\oddsidemargin 0.40in

\begin{document}
\baselineskip 0.30in

\setcounter{chapter}{4}
\setcounter{page}{24}

\chapter{Experimental Results}
	
	One of the key issues in comparing the methods is coming up
with a set of problems that truly represent the diversity in a binary
input output mapping.  It is very important to keep in mind that to
the algorithms these problems are not addition or multiplication, but
simply a set of input output mappings.  So to contrast different
problems it is essential to come up with different types of input
output mappings.  Clearly, given a set of inputs and outputs there is
a finite set of input output mappings to choose from.  The individual
binary encoded problems can be viewed as a rule which defines the
mapping.

	Logistically it is important to come up with a reasonable
number of input bits for starters.  The number of input bits should be
a number where back propagation can be tested quickly and easily
without having to wait a week for each experiment to be run.  Since
statistics is an important factor in determining the comparisons of
learning algorithms, running four tests that take four weeks is
certainly not as beneficial as running one thousand tests that take
several minutes, or one hundred tests that take several hours.  An
issue to sort out is how these problems scale, so it can be determined
which problem size is best representative for determining accurate
statistics in a reasonable amount of time.

	{\it What is a good problem} ?  A problem where back
propagation or herbie is able to learn the input output mapping
perfectly on greater than ninety percent of the initialized runs.  It
must be pointed out that in back propagation for a certain class of
problems each run is very dependent upon the initial set of random
weights.  So if a run is done ten times it may only converge on five
of the runs.  Since we are interested in studying generalization, the
problem must be stable.  Where stability is defined by the above
definition of a good problem.  It has been determined that four to
five input bits is a reasonable number of bits to use in running
comparison experiments.

\section{The Parity Problem}

	The parity problem has been studied for a long time by
researchers who were interested in getting neural networks to learn a
simple input output mapping.  In the PDP series, McClelland and
Rumelhart [1986] point out that although perceptrons were not able to
learn the parity problem, that networks with hidden units were
successful at this supposedly simple task.  Rumelhart and McClelland
go on for several pages showing all the different permutations of the
parity problem.  They experiment with changing the learning rate and
the number of hidden units and show that it still learns but at
different rates.

	Many people who study learning algorithms use the parity
problem as one of their first examples to test that their computer
program is indeed working.  Numerous different studies have
re-analyzed this problem and looked at number of hidden units,
learning rates, etc...  Also, researchers have analyzed hundreds of
different problems to see if back propagation could learn the input
output mapping and usually the conclusion of the study is that indeed
it can or can not learn a particular mapping.

	However, the key point that this paper is looking at is how
well can the algorithm generalize for a particular problem.  It just
so happens, that for the most popular problem studied, {\it parity},
that no generalization is possible.  The following several paragraphs
outlines the method of experiment.

\subsection{Method of Experiment}

	The parity problem studied has five input units, six hidden
units and one output unit.  Since there are five input units there are
thirty two elements in the training set.  The value of the output unit
is derived by simply looking at the number of ones in the input
string.  If there is an even number of ones the output is zero, and if
there is an odd number of ones the output is a one.  So after
inspecting all thirty two elements in the training set half the
outputs are ones and half the outputs are zero.

	The starting point is to train the network on all thirty two
elements in the training set just to make sure that the network can
learn the input output mapping.  The same is true for the local linear
methods.  As suspected, the network learns the input output mapping
perfectly.  Most of the research in the area of learning algorithms
stops at this point.  People will point out the average number of
iterations it took for a particular problem to converge and the
minimum number of hidden units needed to solve the problem.  As
outlined earlier, generalization is the process of training on a
subset of the learning set, and then seeing how well the elements not
in the training set can be learned.

	For the parity problem there are thirty two elements in the
training set.  The generalization experiment is simply remove one of
the thirty two elements	and train the network on thirty one elements.
After the network has learned the input output mapping for the thirty
one elements, then save the network of weights.  This network of
weights defines the multidimensional surface that has been built in
order to recognize thirty one different input output pairs.  This set
of input output pairs defines the function parity in the mathematical
sense.  If the neural network, function, or multidimensional surface
is indeed a good representation of the idea which is called parity;
then when the network is presented with an example that it has never
seen before, {\it ie. the input}, then it should pop out the correct
output value.  It just so happens that because of the nature of the
problem, a neural network or any surface fitting algorithm will not be
able to generalize at all on the parity problem.

	This result has been demonstrated experimentally in the
following way.  Thirty two independent experiments were run on each of
the learning techniques.  For example, with back propagation the
network was trained on all elements in the training set except the
first example [0 0 0 0 0 yields 0].  After the network was trained and
learning completed, then (0 0 0 0 0) was input into the network.
However, instead of a zero on the output a different number was
output.  It therefore missed on this example.  The second experiment
was run and the network was trained on every thing except [0 0 0 0 1
yields 1].  Again, after learning all input output pairs except the
above one, then (0 0 0 0 1) was input to the network and once again a
number other than (1) popped out instead of the expected value one.
This sequence of events was repeated thirty more times, each time the
wrong answer popped out.  The conclusion being that it is impossible
to do generalization on the parity problem.

	The reason why generalization is impossible for parity can be
understood by thinking about the idea of surface fitting.  All of the
problems investigated in this paper are represented on the boolean
hypercube.  The dimension of the hypercube is the number of dimensions
in the input space.  For the parity problem there are five input bits
and therefore the hypercube has five dimensions.  A five dimensional
hypercube has thirty two vertices, hence the number of elements in a
complete training set.

	In order to abstractly think about fitting surfaces through
points on a hypercube one must think more a long the lines of the
local linear method as opposed to the neural network.  Both the local
linear methods and back propagation are fitting high dimensional
surfaces to sets of data points.  This has already been demonstrated
empirically with the parity problem since neither method was able to
generalize at all.

	Having the power to understand what the local linear method is
doing gives one a very nice intuitive feeling of why generalization on
the boolean hypercube for the parity problem is impossible.  The
intuitive idea is that given a particular point on the boolean
hypercube the nearest neighbors to that particular point all have a
different parity than the point in question.  Therefore, one can never
get the correct parity to a problem given one removal.  The
following example depicts the local linear method.

 Do all four (4) parity examples...

\begin{center}
\begin{tabular}{llll}
\multicolumn{1}{c}{Case} &
\multicolumn{2}{c}{Input} &
\multicolumn{1}{c}{Output} \\ 
1. & 0 & 0 & 0 \\
2. & 0 & 1 & 1 \\
3. & 1 & 0 & 1 \\
4. & 1 & 1 & 0 
\end{tabular}
\end{center}


 {\bf Test One,  Test for example 4}

	$0x + 0y + z = 0$

	$0x + 1y + z = 1$      

        $1x + 0y + z = 1$

        $x = 1$,  $y = 1$, $z = 0$

	SO, plugging into example 4

	$1x + 1y + z = 2$     

 Two is not the correct answer, zero is the correct answer.

 {\bf Test Two,  Test for example 3}

	$0x + 0y + z = 0$

	$0x + 1y + z = 1$
      
	$1x + 1y + z = 0$

	$x = -1$,  $y = 1$,  $z = 0$

        SO, plugging into example 3

	$1x + 0y + z = -1$    

 Negative one is not the correct answer, one is the correct answer.

 {\bf Test Three, Test for example 2}

	$0x + 0y + z = 0$

	$1x + 0y + z = 1$
     
	$1x + 1y + z = 0$

	$x = 1$, $y = -1$,  $z = 0$

	SO, plugging into example 2

	$0x + 1y + z = -1$    

 Negative one is not the correct answer, one is the correct answer.

 {\bf Test Four, Test for example 1}

	$0x + 1y + z = 1$

	$1x + 0y + z = 1$
    
	$1x + 1y + z = 0$

	$x = -1$,  $y = -1$,  $z = 2$

	SO, plugging into example 1

	$0x + 0y + z = 2$     

 Two is not the correct answer, zero is the correct answer.

	Although these answers are incorrect, one could argue that a
special little device or piece of software could output a {\it zero}
when it sees a {\it two}, and output a {\it one} when it sees a {\it
negative one}.  This is correct for the parity problem, but not for
other problems.  Since these algorithms are general purpose and not
specific to a particular problem, special machinery to decipher each
individual problem is not feasible.

	Note, I compared these results against the actual herbie code
and got the same results.  This is just one more example to show how
the local linear method actually works in detail.  As you can see, the
local linear method will never be able to get the correct answer.  One
can equate the back propagation algorithm to this problem to
extrapolate the intuition of what is going on with the local neural
network. 

\section{The Encoding Problem}

	In the case of the encoding problem {\it four} bits is indeed
a reasonable number of input bits because it learns the function
perfectly and generalizes perfectly both with one and two removals.
However, {\it three} bits is not enough due to the fact that
generalization is not performed as well on one and two removals.

	Once again, the underlying theme through out this paper is
that both back propagation and local linear methods perform about the
same when it comes to generalization on a set of simple binary encoded
problems.  It was certainly clear in the parity problem that back
propagation and local linear methods performed exactly the same since
neither technique was able to generalize at all.

	The other main theme of this paper is that both techniques are
doing surface fitting.  Intuitively, it is fairly simple to see that
the local linear method is doing surface fitting.  The algorithm is
fitting a multidimensional surface called a simplex through a set of
points which are the nearest neighbors to the point in question.  By
fitting a surface to a set of points, one is able to interpolate to
other areas or sections of the surface whose domain is not part of the
initial set of points.  To make this point clearer, a plane is defined
by three points in space.  Once you have a visual image of the plane,
you can move to infinite sections of that plane which are not part of
the original three points.  It is the ability to move around on the
plane which gives one a sense of surface fitting.

	Because the back propagation algorithm deals with finding a
set of weights in a graph or neural network, it is not quite as clear
that this is a form of surface fitting even though mathematically you
are indeed doing gradient descent.  Since the surface fitting idea is
not as intuitive with back prop, one can use the comparative
experimental results in this paper to further justify that these
algorithms are very similar.

	The experimental results for the encoding and binary
complement problems are identical for both local linear methods and
the neural networks.  The encoding problem is very easy to explain.
Given a five bit input string (1 0 1 1 0), then output the same exact
string (1 0 1 1 0).  The binary complement problem is just the
opposite.  Given a five bit input string (1 0 1 1 0), then output the
exact opposite string.  Although this problem appears to be very
simple to a human being, the generalization results are good, but
not excellent.

\subsection{Experimental Results for One Removal}

	In this paper, two different size input strings were analyzed
for the same two problems, namely encoding and binary complement.  Both
three bit input strings and four bit input strings were tested.  Two
different size input strings were tested in order to contrast the
results between three bit inputs and four bit inputs.  The first test
dealt with one removal of the training set.  So just like in the
parity problem, the network was trained with all elements in the
training set except one.  For the three bit input string the test was
run eight times, and for the four bit input string the test was run
sixteen times, once for each removal.  The results of this testing was
perfect generalization.  In other words, the input output pair that
was not part of the training set was tested with the saved network,
and each time it popped out the correct answer.  This is true for both
the three bit and four bit input string; the encoding and
binary complement problems; and the local linear methods and back
propagation.

\subsection{Experimental Results for Two Removals}

	Another complete set of tests was run for two removals.  In
the problem where there are only eight test cases and three input bits
the complete set of removals only encompasses twenty eight sequences.
Again, this is derived from $8 * (8 - 1) / 2$.  However, in the case
of sixteen test cases and four input bits there are 120 possible
sequences of removals.  Due to computer resources, only a subset of
this group is chosen.  Twenty eight removals were chosen for both
experiments.

	The results of the runs showed that almost perfect
generalization took place in the case of sixteen test cases, but only
a sixty percent generalization rate occurred for the test with eight
test cases.  The reason for this is clear.  When analyzing sixteen
test cases two removals is only 12.5 percent of the total information,
whereas when analyzing eight test cases two removals is twice the
amount of information removed or 25 percent.

	These results held up pretty well as the experiment continued
for the set of sixteen test cases.  When four removals where done on
the set of sixteen tests cases, {\it twenty five percent}, there was
about a fifty percent generalization rate.  The final experiment
involved eight removals out of sixteen test cases.  It was at this
point that no generalization occurred.  Since half the information was
missing, the network was not able to do any generalization.

	This problem supports the thesis that generalization is the
same for both back propagation and local linear methods.  Another nice
result is that this is true for both the encoding problem and the
binary complement problems.  In fact, the results for the encoding
problem and binary complement problems are almost identical.  From an
experimental point of view, this shows that both problems fall into
the same class of ideas that surface fitting methods can solve.

	From the point of view of the neural network, the network
simply has to adjust itself to let the inputs pass through the network
and reproduce the inputs on the output side.  The values of the
outputs on the hidden nodes are the same as the input nodes too.  If
you want to abstractly think of very simple functions, the function $y
= x$ is the same type of idea.  The multidimensional surface acts just
like the function $y = x$ when doing the encoding problem.  The binary
complement problem is very similar.  Taking this a step further, it is
clear that from a learning point of view, reproducing the input is
certainly an easier cognitive task than the parity problem; although
this is certainly beyond the scope of this research.

\section{The Count Problem}
	
	The count problem counts the number of bits in the input
string and outputs the number in a binary encoded output.  This
problem appears to be very straight forward for a human being;
however, for a neural network the problem is extremely difficult as
the generalization results are very unsatisfactory.  In both the case
of back prop and the local linear methods no generalization is
possible even in the simplest case of one removal.

	The generalization score is unsatisfactory for obvious
reasons.  In this particular problem, the discrete space consists of
points which are independent and therefore have no relationship to
each other.  For example, when counting bits in a string, the bit
position next to the one in question has no bearing on whether or not
that particular position will be a one or a zero.  Contrast that with
a digitized image of the letter {\it A}.  In this image, the points
close by are very significant and help the person viewing the object
decipher exactly what he or she is looking at.

\section{Problems with Statistical Analysis}

	When doing generalization with two removals, a problem in
statistics presents itself and it is not real clear how to deal with
it.  One removal is a simple statistical problem.  If you have a five
bit input string you simply do thirty two tests one for each removal.
However, when doing two removals the number of possible removals
increases quadratically.  For example, on a two bit input string the
following rule applies.

\begin{center}
\begin{tabular}{lllll}
\multicolumn{1}{c}{Case} &
\multicolumn{2}{c}{Input} &
\multicolumn{2}{c}{Output} \\
0. & 0 & 0 & 0 & 0 \\
1. & 0 & 1 & 0 & 1 \\
2. & 1 & 0 & 1 & 0 \\
3. & 1 & 1 & 1 & 1 
\end{tabular}
\end{center}

	The number of possible two removals is  $n * (n - 1) / 2$
where $n = 4$, the number of elements in the training set.  So, in this
case the number of possible removals is $4 * (4 - 1) / 2 = 6$.  They are
(01, 02, 03, 12, 13, 23)...

	However for three bit input strings.

\begin{center}
\begin{tabular}{lllllll}
\multicolumn{1}{c}{Case} &
\multicolumn{3}{c}{Input} &
\multicolumn{3}{c}{Output} \\
0. & 0 & 0 & 0 & 0 & 0 & 0 \\
1. & 0 & 0 & 1 & 0 & 0 & 1 \\
2. & 0 & 1 & 0 & 0 & 1 & 0 \\
3. & 0 & 1 & 1 & 0 & 1 & 1 \\
4. & 1 & 0 & 0 & 1 & 0 & 0 \\
5. & 1 & 0 & 1 & 1 & 0 & 1 \\
6. & 1 & 1 & 0 & 1 & 1 & 0 \\
7. & 1 & 1 & 1 & 1 & 1 & 1 
\end{tabular}
\end{center}

	In this particular case $n = 8$ and the number of possible
removals is $8 * (8 - 1) / 2 = 28$.  They are (01, 02, 03, 04, 05, 06,
07, 12, 13, 14, 15, 16, 17, 23, 24, 25, 26, 27, 34, 35, 36, 37, 45,
46, 47, 56, 57, 67)...

	Clearly, it is not feasible to do all the cases of two
removals especially when you are dealing with input strings of size
six or greater.  So what one has to do is take a subset of this larger
set of removals.  One way to take a subset would be to randomly select
a fixed number of removals from the overall set.  Although at first
glance this appears to be a reasonable solution, in practicality it is
only marginal.  The reason being that depending on which removals one
selects, the results can be very different due to the fact that
certain elements on the hypercube close together cause different
effects than elements farther apart.  When you start to scale these
problems and the hypercubes get large, the statistics for one random
set of removals is not statistically significant.  This leads to a set
of a set of random removals.

\end{document}
