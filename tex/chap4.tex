\documentstyle[12pt]{report}
\parskip 0.20in
\textheight 8.75in
\textwidth 6.0in
\topmargin -0.25in
\oddsidemargin 0.40in

\begin{document}
\baselineskip 0.30in

\setcounter{chapter}{3}
\setcounter{page}{19}

\chapter{The Binary Encoded Problems}	

	Learning algorithms are used on two major types of problems.
The two types of problems are continuous and discrete.  In this paper only
simple discrete problems are studied.

 	Continuous problems usually fall into the category of prediction
of a time series.  For example, if one has daily stock prices of
I.B.M. for the first half of a year, then one could train the learning
algorithm using the first six months data.  After the training has
been completed, then one could try to predict the next week of stock
prices for I.B.M..  Similar experiments could be set up to predict sun
spots in order to warn NASA of the next big sun spot.

	Although these may sound like useful, profitable experiments,
in reality these time series are extremely difficult to predict.  As
with any real world data set, there is always a certain amount of
observational noise.  Since one can not determine easily how much
noise is in a data set, it is very difficult to analyze the results.
One of the important variables in local linear techniques is the
dimensionality of the problem.  With real world data, choosing a
dimension is extremely arbitrary.  One of the few approaches to
finding the dimension of a time series is by calculating the fractal
dimension.  The algorithm for calculating fractal dimension of a time
series is not yet well understood, see Barnsley [1988]. 

	For the above stated reasons, simple discrete problems are
much easier to use for testing learning algorithms.  The discrete
learning problems chosen in this paper have no noise in the data.
Also, since the domains are simple to understand, dimensions of the
data sets in this paper are easy to figure out.  By avoiding these
uncertainties, the reliability of the results are much more accurate.

	There is however, still one major obstacle with discrete
problems and that is the task of encoding the problem so that the
learning algorithm can accurately represent what is going on.  

	For example, if the algorithm is acting on a data set of
English words, should the words be represented by unique symbols or
the individual letters of the word.  If the algorithm is learning to
distinguish objects in a room, what representation should be used so
that the learning technique can understand many different objects.
Finally, what if the algorithm is trying to learn how to pick up a
full glass of water; how is this accurately represented ?  As one can
quickly see, the problem of encoding a discrete task so that the
learning algorithm can act on the data set is extremely non trivial.

	Because the purpose of this paper is mainly to compare two
types of learning algorithms, and not the optimal way to encode or
represent a problem; very simple binary encoded problems were chosen.
Once these techniques are more widely understood both mathematically
and fundamentally, then harder problems should indeed be attempted.
But for now, simple problems is the best way to get the point across
without adding more levels of abstraction, or more significantly
complication. 

\section{The Actual Problems}

	Simple binary encoded problems are convenient for analyzing
empirical results.  By using simple problems, people can work through
the examples by hand and prove to themselves exactly what these local
linear techniques are doing mathematically.  Since the focus of this
paper is to compare and contrast two learning algorithms, then the
problems should simply serve as a means to an end.  Future research in
this area could start to address more difficult problems; however, for
now the local linear ideas should be understood with simple binary
encoded discrete problems.

\subsection{Three Bit Addition}

	Simple three bit addition

	$0\  1\  0\   +\   1\  0\  0\   \Longrightarrow  0\ 1\  1\  0$

        $1\  0\  1\   +\   0\  1\  1\   \Longrightarrow  1\ 0\  0\  0$

	Note, the set will contain all permutations of three bit
addition.  So in total there are 64 elements in the training set.
Note, that when you add two three bit numbers the max you can have for
an output is a four bit number.  So, the input dimension to this
problem is six and the output dimension is four.

\subsection{Three Bit Multiplication}
	
	Simple three bit multiplication

	$0\  1\  0\   *\  1\  0\  0\   \Longrightarrow  0\ 0\ 1\ 0\ 0\ 0$

        $1\  1\  1\   *\  1\  1\  0\   \Longrightarrow  1\ 0\ 1\ 0\ 1\ 0$

	Note, the set will contain all permutations of three bit
multiplication.  So in total there are 64 elements in the training set.
Note, that when you multiply two three bit numbers the max you can
have for an output is a six bit number.  So, the input dimension to
this problem is six and the output dimension is six.
	

\subsection{The Count Problem}

	Given a string of 1's and 0's simply count up the number of
1's in the string and output that particular number.

	For example,
		
		$0\  1\  0\  1\  0\  0\  \Longrightarrow \ 0\  1\  0$

                $1\  1\  1\  0\  0\  0\  \Longrightarrow \ 0\  1\  1$

                $1\  0\  0\  0\  1\  0\  \Longrightarrow \ 0\  1\  0$

\subsection{The Parity Problem}

	The parity problem determines whether there is an even or odd
number of one's in the bit string.  If there are even number of ones
it outputs a zero, and if there are an odd number of ones it outputs a
one. 

	For example,

		$0\  1\  0\  1\  0\  0\  \Longrightarrow \ 0$

                $1\  1\  1\  0\  0\  0\  \Longrightarrow \ 1$

                $1\  0\  0\  0\  1\  0\  \Longrightarrow \ 0$

\subsection{The Encoding Problem}

	The encoding problem simply reproduces the input string on the
output string.

	For example,

	       $0\ 1\ 0\ \Longrightarrow 0\ 1\ 0$

	       $0\ 1\ 1\ \Longrightarrow 0\ 1\ 1$

	       $1\ 0\ 0\ \Longrightarrow 1\ 0\ 0$

\subsection{The Binary Complement Problem}

	The binary complement problem simply outputs the exact opposite
of the input string.

	For example,

	       $0\ 1\ 0\ \Longrightarrow 1\ 0\ 1$

	       $0\ 1\ 1\ \Longrightarrow 1\ 0\ 0$

	       $1\ 0\ 0\ \Longrightarrow 0\ 1\ 1$

\section{The Experiments}

	Note the number of bits in the actual experiments may slightly
differ from the above examples.  However, the significance of this
section was to outline the definition of each problem.  

\end{document}
