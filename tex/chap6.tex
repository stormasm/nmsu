\documentstyle[12pt]{report}
\parskip 0.20in
\textheight 8.75in
\textwidth 6.0in
\topmargin -0.25in
\oddsidemargin 0.40in

\begin{document}
\baselineskip 0.30in

\setcounter{chapter}{5}
\setcounter{page}{37}

\chapter{Time Comparisons}

	In this chapter two types of time comparisons are explained.
The first deals with the theoretical issues of space and time when
comparing the two methods.  The second section looks at actual timing
studies.

\section{Space and Time Trade Off}

	One of the most common criteria for comparing algorithms is
the amount of space and time that the algorithm consumes.  In order to
compare back propagation and the local linear methods an analysis of
these two factors is examined.  This argument could be coined {\it the
space time trade off}.

	Back propagation has two stages, the first stage is training
and the second stage is delivery.  In order to better understand these
ideas an analogy to a real world environment will be explained.  Given
a set of one million input output pairs it might take back propagation
one month to learn the complete mapping.  Once the mapping is learned
and stored in the network the training stage is completed.  The
delivery phase simply entails distributing the network to researchers
interested in doing generalization on this particular function.  It
might also be possible that in the future the network will be
delivered in silicon.  In either case, once the network has been
delivered, running the algorithm is simple and very fast.  The amount
of time it takes to propagate an input value through a network and
produce an output value is around one second in software and much
quicker in hardware.  Also note that the previous million training
values are no longer needed for the computation.  The space and time
needed to generate an output value once training is completed is
independent of the number of items in the training set.

	This is not the case for local linear methods.  There is no
training phase in this algorithm.  Every single time an output value
is requested, the algorithm has to run on all million points to
generate an answer.  Clearly this takes up a lot of space relative to
the delivery phase of back propagation.

	From an intial point of view, if one just wants to see if a
learning algorithm can generalize a particular function the local linear
method is the obvious choice.  Because with a million points, it is a
lot quicker to run the local linear method and come up with an answer
in one hour, than wait one month for an answer from back propagation.
If on the average local linear is three orders of magnitude quicker
than back propagation, then the one hour / one month analogy is fairly
accurate.  

	In certain instances, it might be better to use back
propagation especially if one is going to be doing a lot of
generalizing in the delivery phase.  After running back propgation for
one month and the training is complete, then it will not take more
than one second to produce an output value given any input.  If you
need to run local linear more than one thousand times, then back
propagation will be faster and more efficient for both space and time.

\section{Timing Studies}

	The local linear method is somewhere between 3 to 4 orders of
magnitude faster than back propagation.  

	A comparative study has been done to see how long it takes for
back propagation to learn an input output mapping as the size of the
problem or network is increased.  The amount of time it takes to learn
the parity problem for an input size of eight is 930 minutes of cpu
time on the Sparc Station.  This is a very large number considering
the fact that the problem was run on the fastest scientific work
station on the market today.  The Sparc Station is bench marked at
approximately twelve MIPS.  Considering that eight input bits is not
that large some estimates were done on problems ranging in the area of
twenty to thirty input bits.  These estimates were based on the
average number of iterations it takes for these problems to converge.
Clearly one could not tie up a Sun Station for one month just to see
how long it takes back prop to learn the parity problem with thirty
input bits.
	
	The point of this study is just to show that although back
prop may be a popular algorithm, it certainly is not the optimal
algorithm for doing generalization.  Especially since the local linear
methods perform as well as back prop on the discrete problems chosen
in this paper.

\end{document}
